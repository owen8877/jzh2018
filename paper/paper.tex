\documentclass[UTF8]{ctexart}

\CTEXsetup[format={\Large\bfseries}]{section}  
\usepackage{amsmath, amsfonts, amsthm, bm, enumerate, ulem}
\usepackage{epstopdf, graphicx}
\usepackage{algorithm}  
\usepackage{algorithmicx}  
\usepackage{algpseudocode}
\usepackage{booktabs}
\usepackage{mathrsfs}
\usepackage{geometry}
\geometry{scale = 0.8}
\usepackage[colorlinks, linkcolor=red, anchorcolor = blue, citecolor=green]{hyperref}

\floatname{algorithm}{算法}
\renewcommand{\algorithmicrequire}{\textbf{输入:}} 
\renewcommand{\algorithmicensure}{\textbf{输出:}}
\usepackage{listings}
\usepackage{xcolor}
\lstset{
	numbers=left, 
	numberstyle= \tiny, 
	keywordstyle= \color{ blue!70},
	commentstyle= \color{red!50!green!50!blue!50}, 
	frame=shadowbox, % 阴影效果
	rulesepcolor= \color{ red!20!green!20!blue!20} ,
	escapeinside=``, % 英文分号中可写入中文
	xleftmargin=2em,xrightmargin=2em, aboveskip=1em,
	framexleftmargin=2em
} 

\begin{document}
	\setcounter{footnote}{1}
	\title{基于跳过程与统计学习的普适排行榜类产品度量模型}
	\author{程晨\footnote{School of Mathematical Sciences, Peking University, Beijing 100871, China. Email address:
			\href{mailto:moriartycc@pku.edu.cn}{moriartycc@pku.edu.cn}, ID: 1500010714} \quad 陈子恒\footnote{School of Mathematical Sciences, Peking University, Beijing 100871, China. }, \quad 郝天泽\footnote{School of Mathematical Sciences, Peking University, Beijing 100871, China. }}
	\date{}
	\maketitle
	\abstract{
		
		\textbf{关键字:} 
	}
	\newpage
	\tableofcontents
	\newpage
	\section{引言}
	\section{假设}
	\section{无跳过的顺序选择模型——Steam探索队列}
	\subsection{模型概述 \& 符号约定}
	\subsection{度量模型}
	\subsection{统计推断 \& 优化算法}
	\subsection{实验结果}
	\section{带跳过的顺序选择模型——美团外卖}
	\subsection{模型修正 \& 符号约定}
	\subsection{度量模型 \& 统计推断 \& 优化算法}
	\subsection{实验结果}
	\section{带跳过的投票模型——豆瓣、知乎推荐}
	\subsection{模型修正 \& 符号约定}
	\subsection{度量模型 \& 统计推断 \& 优化算法}
	\subsection{实验结果}
	\section{实际产品评价}
	\subsection{两种著名的排行榜产品}
	\subsection{不同情形下的实验比较}
	\section{由此启发得到的基于优化的排行榜算法}
	\section{总结}
	\bibliography{bib}
\end{document}
		
		 